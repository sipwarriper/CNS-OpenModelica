\documentclass[11pt,a4paper,twoside]{report}
  \usepackage{a4wide}
  \usepackage{epsfig}
  \usepackage{amsmath}
  \usepackage{tabu}
  \usepackage{amsfonts}
  \usepackage{latexsym}
  \usepackage[utf8]{inputenc}
  \usepackage{listings}
  \usepackage{color}
  \usepackage{titlesec}    
  \usepackage{enumitem}
  \usepackage[catalan]{babel}
  \usepackage{newunicodechar}
  \usepackage{graphicx}
  \usepackage{subcaption}
  \usepackage{float}
  \usepackage[numbered,framed]{matlab-prettifier}
  \usepackage{xcolor}
  \usepackage{pgf, tikz}
  \usepackage{listings}
  %% listings-modelica.cfg
%% Copyright 2014 Martin Sjoelund, Dietmar Winkler
%
% This work may be distributed and/or modified under the
% conditions of the LaTeX Project Public License, either version 1.3
% of this license or (at your option) any later version.
% The latest version of this license is in
%   http://www.latex-project.org/lppl.txt
% and version 1.3 or later is part of all distributions of LaTeX
% version 2005/12/01 or later.
%
% This work has the LPPL maintenance status `maintained'.
%
% The Current Maintainer of this work is Dietmar Winkler
%
% Code repository https://github.com/modelica-tools/listings-modelica
%
% This work consists of the file listings-modelica.cfg

\lstdefinelanguage{modelica}
{
  morekeywords=[1]{
    algorithm,and,annotation,as,assert,block,break,case,class,connect,connector,
    constant,constrainedby,der,discrete,each,else,elseif,elsewhen,encapsulated,
    end,enumeration,equality,equation,expandable,extends,external,failure,final,
    flow,for,function,guard,if,import,in,initial,inner,input,List,local,loop,
    match,matchcontinue,model,not,operator,Option,or,outer,output,package,parameter,
    partial,protected,public,record,redeclare,replaceable,return,stream,
    subtypeof,then,Tuple,type,uniontype,when,while},
  morekeywords=[2]{true, false},
  % Do not make true,false keywords because fn(true,x, false ) shows up as fn(true,x, *false*)
  morekeywords=[3]{optimization,constraint}, % Optimica keywords
  morekeywords=[4]{objective,startTime,finalTime,initialGuess},
  sensitive=true,
  comment=[l]//,
  morecomment=[s]{/*}{*/},
  alsodigit={.,-},
  morestring=[b]',
  morestring=[b]",
}[keywords,comments,strings]

\definecolor{keywordcolor1}{rgb}{0,0,.4}
\definecolor{keywordcolor2}{rgb}{.90,0,0}
\definecolor{keywordcolor3}{rgb}{.4,0,.8}
\definecolor{keywordcolor4}{rgb}{0.5,0,0.5}
\definecolor{stringcolor}{rgb}{0.133,0.545,0.133}
% \definecolor{listingbgcolor}{rgb}{0.95,0.95,0.95}

\lstset{
  breaklines=true,
  language=modelica,
  basicstyle=\ttfamily,
  keywordstyle=[1]\color{keywordcolor1}\bfseries,
  keywordstyle=[2]\color{keywordcolor2},
  keywordstyle=[3]\color{keywordcolor3}\bfseries,
  keywordstyle=[4]\color{keywordcolor4},
  stringstyle=\color{stringcolor},
%  backgroundcolor=\color{listingbgcolor},
  framexleftmargin=5pt,
  xleftmargin=5pt,
  xrightmargin=5pt,
  showstringspaces=false
}

\newcommand{\code}[1]{\lstinline|#1|}
\newcommand{\modelica}[1]{\lstinline[language=modelica]|#1|}

  \usetikzlibrary{arrows, automata, positioning, datavisualization, datavisualization.formats.functions}
  
\setcounter{tocdepth}{4}
\setcounter{secnumdepth}{4}
  
\newunicodechar{Ŀ}{\L.}
\newunicodechar{ŀ}{\l.}

\definecolor{dkgreen}{rgb}{0,0.6,0}
\definecolor{gray}{rgb}{0.5,0.5,0.5}
\definecolor{mauve}{rgb}{0.58,0,0.82}

\lstset{frame=tb,
language=Matlab,
aboveskip=3mm,
belowskip=3mm,
showstringspaces=false,
columns=flexible,
basicstyle={\small\ttfamily},
numbers=none,
numberstyle=\tiny\color{gray},
keywordstyle=\color{blue},
commentstyle=\color{dkgreen},
stringstyle=\color{mauve},
breaklines=true,
breakatwhitespace=true,
tabsize=3,
extendedchars=true,
literate={á}{{\'a}}1 {à}{{\`a}}1 {ã}{{\~a}}1 {é}{{\'e}}1 {è}{{\`e}}1 {í}{{\'i}}1 {ï}{{\"i}}1 {ó}{{\'o}}1 {ò}{{\`o}}1 {ú}{{\'u}}1 {ü}{{\"u}}1 {ç}{{\c{c}}}1
			{Á}{{\'A}}1 {À}{{\`A}}1 {Ã}{{\~A}}1 {É}{{\'E}}1 {È}{{\`E}}1 {Í}{{\'I}}1 {Ï}{{\"I}}1 {Ó}{{\'O}}1 {Ò}{{\`O}}1 {Ú}{{\'U}}1 {Ü}{{\"U}}1 {Ç}{{\c{C}}}1
}

\usepackage{hyperref}
\hypersetup{
  colorlinks=false, %set true if you want colored links
  linktoc=all,     %set to all if you want both sections and subsections linked
  linkcolor=blue,  %choose some color if you want links to stand out
}

\setlength{\footskip}{50pt}
\setlength{\parindent}{0cm} \setlength{\oddsidemargin}{-0.5cm} \setlength{\evensidemargin}{-0.5cm}
\setlength{\textwidth}{17cm} \setlength{\textheight}{23cm} \setlength{\topmargin}{-1.5cm} \addtolength{\parskip}{2ex}
\setlength{\headsep}{1.5cm}

\lstset{language = modelica,
        basicstyle=\fontsize{9pt}{10.5pt}\ttfamily,
        backgroundcolor = \color{white}}

\renewcommand{\contentsname}{Continguts}
\setcounter{chapter}{0}

\begin{document}

\title{Simulació d'un sistema de Service Desk}
\author{Marc Cané, Ismael El Habri, Lluís Trilla}
\date{12 de desembre de 2018}
\maketitle

\tableofcontents

\chapter{Exercicis proposats}

\section{Exercici 1}

Per a fer el model hem dissenyat diferents models intermitjos, que ens serviran per simular cada fase del procés:
\begin{itemize}
  \item \textbf{ServiceDesk}: Model que simula tot el sistema de service desk de la empresa.
  \item \textbf{Empresa}: Model que simula la generació d'incidències de l'empresa.
  \item \textbf{Resolució}: Model que simula la resolució d'incidències.
  \item \textbf{UnificadorSolucionades}: Model que rep totes les incidències resoltes i les unifica.
  \item \textbf{incidencies}: Classe connector per transmetre incidències 
\end{itemize}

\subsection{Connector Incidencies}

\lstinputlisting[language=modelica]{../Incidencies.mo}

Aquesta classe no te cap secret, és de tipus connector i té un element Real \texttt{output} amb les incidències que es van passant.

\subsection{Model Empresa}

\lstinputlisting[language=modelica]{../Empresa.mo}
Passem per paràmetre al instanciar el rati d'incidències, el nombre de treballadors i el rati de reopertures. Té dos connectors de incidències, un de sortida (generades) i un d'entrada (tancades).

\subsection{Model Resolucio}

\lstinputlisting[language=modelica]{../Resolucio.mo}

Model al qual li passem per paràmetre la formació i el màxim de resolucions que pot fer cada persona per hora. Té a més, tres connectors d'Incidències, les d'entrada, les tancades, i les que s'envien al següent nivell.
Aquest model l'hem fet de forma que no quedin incidències pendents cada hora, ficant com a variable el nombre de treballadors. 
Ficant la fórmula pertinent (el que vindrien a ser les incidències pendents) igualada a 0, ens fa el càlcul al fer la simulació.

\subsection{Model UnficadorSolucionades}
\lstinputlisting[language=modelica]{../UnificadorSolucionades.mo}

Model de suport amb tres connectors d'Incidències d'entrada i un de sortida, que ens suma el valor dels tres connectors d'entrada.

\subsection{Model ServiceDesk}
\lstinputlisting[language=modelica]{../ServiceDesk.mo}

Aquest model es el model el qual fa la simulació completa. Té tres objectes Resolucio (un per cada nivell de formació), un UnificadorSolucionades i un Empresa. 
Aquests al instanciar-se sels hi ha de passar els paràmetres corresponents. Després, al apartat d'equacions el que fem és connectar els connectors de cada classe seguint el següent dibuix:



\begin{figure}%[H]
  \centering
  \begin{tikzpicture}[
    > = stealth, % arrow head style
    shorten > = 1pt, % don't touch arrow head to node
    auto,
    node distance = 3cm, % distance between nodes
    semithick, % line style
    state/.style={circle, draw, minimum size=5cm}
  ]
  
  \tikzstyle{every state}=[
    draw = black,
    thick,
    fill = white,
    minimum size = 4mm
  ]
  
  \node[state] (v1) {Empresa};
  \node[state] (v2) [right = 4cm of v1] {Resolució Nivell 1};
  \node[state] (v3) [below = 3cm of v2] {Resolució Nivell 2 };
  \node[state] (v4) [below = 3cm of v3] {Resolució Nivell 3};
  \node[state] (v5) [left = 4cm of v3] {Unificador Solucionades};
  
  \path[->] (v1) edge node {Generades/entrada} (v2);
  \path[->] (v2) edge node {SeguentNivell/entrada} (v3);
  \path[->] (v3) edge node {SeguentNivell/entrada} (v4);
  \path[->] (v2) edge node {Tancades/n1} (v5);
  \path[->] (v3) edge node {Tancades/n2} (v5);
  \path[->] (v4) edge node {Tancades/n3} (v5);
  \path[->] (v5) edge node {Sortida/tancades} (v1);
  \end{tikzpicture}
\end{figure}

\newpage
\section{Exercici 2}
Els nombre de persones que equilibren el sistema són 13 treballadors de nivell 1, 7 treballadors de nivell 2 i 7 treballadors de nivell 3.
\newline Els treballadors del nivell 1 resolen una mitjana de 5.05 incidències per hora mentre que els de nivell 2 i 3 en resolen 2.53.
En total hem necessitat 27 treballadors.

\section{Exercici 3}
%\frac{10}{13+7+7} = \text{0.37037} 
La mitjana d'incidències resoltes per persona i per dia són 8.89, que equival a 0.37 incidències resoltes per hora.

\section{Exercici 4}
El nou punt d'equilibri després d'afegir 3000 usuaris és 17 treballadors de nivell 1, 9 treballadors de nivell 2 i 9 treballadors de nivell 3.
\newline Es passa de tancar una mitja de 10.1 incidències cada hora a tancar-ne 13.13.  
\newline Els treballadors de nivell 1 resolen ara una mitjana de 6.57 incidències per hora mentre que els de nivell 2 i 3 en resolen 3.28.

\section{Exercici 5}
TODO

\section{Exercici 6}
El nou punt d'equilibri és 13 treballadors de nivell 1, 10 treballadors de nivell 2 i 3 treballadors de nivell 3.
\newline Al augmentar la formació dels treballadors hem passat de necessitar 27 treballadors en el supòsit original a 23.

\section{Exercici 7}
El nou punt d'equilibri és 20 treballadors de nivell 1, 3 treballadors de nivell 2 i 3 treballadors de nivell 3.
\newline Continuem necessitant el mateix nombre de treballadors que en el supòsit anterior.

\section{Exercici 8}
Si augmentem la formació dels usuaris i reduïm el número d'incidències generades a la meitat els usuaris tindràn una taxa d'indicències de 1/2000 incidències generades per persona i hora.
\newline El nou punt d'equilibri és de 7 treballadors de nivell 1, 4 treballadors de nivell 2 i 4 treballadors de nivell 3.
S'ha reduït el nombre de treballadors necessaris de 27 a 15.

\end{document}
