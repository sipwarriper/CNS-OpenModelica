\documentclass[11pt,a4paper,twoside]{report}
  \usepackage{a4wide}
  \usepackage{epsfig}
  \usepackage{amsmath}
  \usepackage{tabu}
  \usepackage{amsfonts}
  \usepackage{latexsym}
  \usepackage[utf8]{inputenc}
  \usepackage{listings}
  \usepackage{color}
  \usepackage{titlesec}    
  \usepackage{enumitem}
  \usepackage[catalan]{babel}
  \usepackage{newunicodechar}
  \usepackage{graphicx}
  \usepackage{subcaption}
  \usepackage{float}
  \usepackage[numbered,framed]{matlab-prettifier}
  \usepackage{xcolor}
  \usepackage{pgf, tikz}
  \usepackage{listings}
  \input{listings-modelica.cfg}
  \usetikzlibrary{arrows, automata, positioning, datavisualization, datavisualization.formats.functions}
  
\setcounter{tocdepth}{4}
\setcounter{secnumdepth}{4}
  
\newunicodechar{Ŀ}{\L.}
\newunicodechar{ŀ}{\l.}

% \titleformat{\chapter}
%   {\normalfont\LARGE\bfseries}{\thechapter}{1em}{}
% \titlespacing*{\chapter}{0pt}{3.5ex plus 1ex minus .2ex}{2.3ex plus .2ex}

\definecolor{dkgreen}{rgb}{0,0.6,0}
\definecolor{gray}{rgb}{0.5,0.5,0.5}
\definecolor{mauve}{rgb}{0.58,0,0.82}

\lstset{frame=tb,
language=Matlab,
aboveskip=3mm,
belowskip=3mm,
showstringspaces=false,
columns=flexible,
basicstyle={\small\ttfamily},
numbers=none,
numberstyle=\tiny\color{gray},
keywordstyle=\color{blue},
commentstyle=\color{dkgreen},
stringstyle=\color{mauve},
breaklines=true,
breakatwhitespace=true,
tabsize=3,
extendedchars=true,
literate={á}{{\'a}}1 {à}{{\`a}}1 {ã}{{\~a}}1 {é}{{\'e}}1 {è}{{\`e}}1 {í}{{\'i}}1 {ï}{{\"i}}1 {ó}{{\'o}}1 {ò}{{\`o}}1 {ú}{{\'u}}1 {ü}{{\"u}}1 {ç}{{\c{c}}}1
			{Á}{{\'A}}1 {À}{{\`A}}1 {Ã}{{\~A}}1 {É}{{\'E}}1 {È}{{\`E}}1 {Í}{{\'I}}1 {Ï}{{\"I}}1 {Ó}{{\'O}}1 {Ò}{{\`O}}1 {Ú}{{\'U}}1 {Ü}{{\"U}}1 {Ç}{{\c{C}}}1
}

\usepackage{hyperref}
\hypersetup{
  colorlinks=false, %set true if you want colored links
  linktoc=all,     %set to all if you want both sections and subsections linked
  linkcolor=blue,  %choose some color if you want links to stand out
}

\newcommand\double[3][10]{%Passantli A i B genera quatre vertexs virtuals A-B-s, A-B-e (per resepresentar una aresta) i B-A-s, B-A-e (per representar l'altre aresta)
  \draw (#2)
    edge [bend left=#1,draw=none]
    coordinate[at start](#2-#3-s)
    coordinate[at end](#2-#3-e)
    (#3)
    edge [bend right=#1,draw=none]
    coordinate[at start](#3-#2-e)
    coordinate[at end](#3-#2-s)
    (#3);
}

\setlength{\footskip}{50pt}
\setlength{\parindent}{0cm} \setlength{\oddsidemargin}{-0.5cm} \setlength{\evensidemargin}{-0.5cm}
\setlength{\textwidth}{17cm} \setlength{\textheight}{23cm} \setlength{\topmargin}{-1.5cm} \addtolength{\parskip}{2ex}
\setlength{\headsep}{1.5cm}

\lstset{language = modelica,
        basicstyle=\fontsize{9pt}{10.5pt}\ttfamily,
        backgroundcolor = \color{white}}

\renewcommand{\contentsname}{Continguts}
\setcounter{chapter}{0}

\begin{document}

\title{Simulació d'un sistema de Service Desk}
\author{Marc Cané, Ismael El Habri, Lluís Trilla}
\date{12 de desembre de 2018}
\maketitle

\tableofcontents

\chapter{Exercicis plantejats}

\section{Exercici 1}

Per a fer el model hem dissenyat diferents models intermitjos, que ens serviran per simular cada fase del procés:
\begin{itemize}
  \item \textbf{ServiceDesk}: Model que simula tot el sistema de service desk de la empresa.
  \item \textbf{Empresa}: Model que simula la generació d'incidències de l'empresa.
  \item \textbf{Formació}: Model que simula la resolució d'incidències.
  \item \textbf{UnificadorSolucionades}: Model que rep totes les incidències resoltes i les unifica.
  \item \textbf{incidencies}: Classe connector per transmetre incidències 
\end{itemize}

\subsection{Connector Incidencies}

\lstinputlisting[language=modelica]{../Incidencies.mo}

Aquesta classe no te cap secret, és de tipus connector i té un element Real \texttt{output} amb les incidències que es van passant.

\subsection{Model Empresa}

\lstinputlisting[language=modelica]{../Empresa.mo}
Passem per paràmetre al instanciar el rati d'incidències, el nombre de treballadors i el rati de reopertures. Té dos connectors de incidències, un de sortida (generades) i un d'entrada (tancades).

\subsection{Model Resolucio}

\lstinputlisting[language=modelica]{../Resolucio.mo}


\lstinputlisting[language=modelica]{../ServiceDesk.mo}

\lstinputlisting[language=modelica]{../UnificadorSolucionades.mo}

\newpage
\section{Exercici 2}
Els resultats que equilibren el sistema són 4 treballadors per el nivell 1, 4 treballadors per el nivell 2 i 2 treballadors per el nivell 3.

\section{Exercici 3}
La mitjana d'incidències resoltes per persona i hora són: 

\end{document}
